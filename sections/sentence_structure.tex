%&latex
\section{Sentence Structure}
\Langname{} has a default word order of verb--subject--object. However, since
nouns are marked for case, the elements can be rearranged for emphasis.

\subsection{Coördination}
In \Langname, multiple nouns, or multiple verbs, can be chained together to
form a single element of a sentence, without any connecting words.

\paragraph{Nouns} Nouns in coördination have two possible meanings. In the
first meaning, called \emph{attributive coördination}, the nouns all refer
to the same thing, providing additional specification or description. In this
case, all the nouns must agree in case and (usually) number. In the second
meaning, called \emph{comitative coördination}, the nouns refer to separate
things, which together fill the role in the sentence.  In this case, the nouns
must agree in case, but not necessarily in number, and may not appear in the
dual number.

\paragraph{Verbs} Verbs in coördination also have two possible meanings. When
the first verb is imperfective, the verbs typically refer to actions or states
which are happening at the same time. This is called \emph{concurrent
    coördination}. Some verbs, such as {\ll n.nd} \qq{to do slowly}, when used
this way, have the effect of modifying the following verb.  When the first verb
is perfective, the verbs typically refer to sequential events, or states which
result from actions. This is called \emph{sequential coördination}.
% TODO: add examples

\subsection{Relative Clauses}
Relative clauses are marked with the conjunction {\ll zəj}, which joins a noun
to a relative clause. In the relative clause, if the head noun fills the role
of subject or direct object, it can be dropped. Otherwise, an appropriate
pronoun is used to fill the gap.

\pex
\a
\begingl
\glpreamble réhkèʔ nájslàr məwhəs zəj jənhón ɦə̀sp. //
\gla r.h @ -kèʔ n´js @ -lár m.wh @ -əs zəj j.n @ -hôn ɦesp //
\glb go -{\sc prs}.3s live -{\sc loc} man -{\sc nom}.s {\sc rel} see -{\sc
pst}.1s yesterday //
\glft The man that I saw yesterday went home.//
\endgl

\a
\begingl
\glpreamble tnəkdéhkèʔ məwhəs zəj trówò hówtjòtwáshə̀ krâ. //
\gla tn.k @ -déh @ -kèʔ m.wh @ -əs zəj tr´ -wò h.wt @ -jò @ -twas @
-hè krâ //
\glb {[be ill]} -{\sc inch} -{\sc prs}.3s man -{\sc nom}.s {\sc rel} give -{\sc
prs}.1s boil -{\sc ret} -{\sc exp} -{\sc acc}.s {\sc pn}.3.{\sc loc}.s //
\glft The man that I gave the tea to got sick. //
\endgl
\xe

