%&latex
\section{Nouns}

Nouns are generally formed by adding a nominalizing ending to a verbal root,
although some roots can be used on their own as nouns.

Nouns decline for three numbers: singular, dual, and plural, and five cases:
nominative, accusative, instrumental, genitive, and locative.
The nominative is only found in animate nouns, and serves as the subject of a
verb.
The accusative serves as the object of a verb.
The instrumental is only found in inanimate nouns, and indicates the means of
an action.
The genitive can be translated as \qq{from} or \qq{of}, and is used for motion
away from, and certain types of possession.
In the dual, or in comitative coördination, the genitive can also be used to
mean \qq{between}.
The locative can be translated as \qq{to} or \qq{at}, and indicates motion
towards, location, other types of possession.
In addition to their standalone meanings, different cases are used as the objects of
prepositions, and verbs may have specific uses for the genitive and locative
cases. The endings are given in \cref{tab:noun endings}.

Because inanimate nouns lack a nominative case, they can't be the subject of a
sentence. Instead, they are placed in the instrumental case, and the impersonal
form of the verb is used:

\ex
\begingl
    \glpreamble kénjoɦə lomlarja məwhtəhjə. //
    \gla kén @ -jo @ -ɦə lomlar @ -ja məwh @ -təh @ -jə //
    \glb separate -{\sc ret} -{\sc prs}.0 mountain -{\sc ins}.s person -{\sc
        col} -{\sc acc}.d //
    \glft The mountain separates the two peoples. //
\endgl
\xe

\begin{table}[h]
\centering
\caption{Noun Endings}
\label{tab:noun endings}
\begin{threeparttable}
\makebox[\linewidth]{
\begin{tabular}{c*{3}{>{\ll}c}}
    \toprule
    Case & \rm Singular & \rm Dual & \rm Plural \\ \midrule
    Nominative   & (ə)s\tnote{\rm 1} & jə  & (ə)h\tnote{\rm 1} \\
    Accusative   & hè\tnote{\rm 2}   & jə  & həh \\
    Instrumental & ja                & jaj & jah \\
    Genitive     & (a)j\tnote{\rm 1} & za  & (a)jh\tnote{\rm 1} \\
    Locative     & ra                & raj & rah \\
    \bottomrule
\end{tabular}
}
\begin{tablenotes}
\item[1] Unlike unlike normal endings, these ending do not start with
    consonants. If they are added to an ending without a final consonant, their
    vowels are dropped.
\item[2] The grave accent indicates that this ending cannot be stressed.
\end{tablenotes}
\end{threeparttable}
\end{table}

\subsection{Pronouns}
\paragraph{Personal Pronouns} There are first and second person pronouns, as
well as several sets of demonstratives that function as third person pronouns.
Since subjects are marked on verbs, the nominative case of the pronouns are
only used when two subjects appear in comitative coördination (e.g. \dq{You and
    I \ldots}). The dual number cannot be used in this type of coördination,
and therefore the personal pronouns lack a nominative in the dual. The forms of
the pronouns are shown on \cref{tab:personal pronouns}.

\begin{table}[h]
\centering
\caption{Personal Pronouns}
\label{tab:personal pronouns}
\begin{tabular}{ll*{3}{>{\ll}c}}
    \toprule
    Person & Case & \rm Singular & \rm Dual & \rm Plural \\ \midrule
    First Person  & Nominative & wəs   & —      & thəh \\
                  & Accusative & weh   & ʔehè   & thəhəh \\
                  & Genitive   & waj   & ʔeza   & thajh \\
                  & Locative   & wəra  & ʔera   & thərah \\ \midrule
    Second Person & Nominative & sáns  & —      & səh \\
                  & Accusative & sánhe & snəj   & səhəh \\
                  & Genitive   & sánaj & sánza  & sánajh \\
                  & Locative   & sánra & sánraj & sánrah \\
    \bottomrule
\end{tabular}
\end{table}

\paragraph{Demonstratives} \Langname{} has three sets of demonstratives. The
first, with a stem of {\ll k.-}, is an animate anaphoric demonstrative, meaning
that it refers to an animate noun (generally a person or animal) that has
already been referred to. The second, with a stem of {\ll to-}, is an
inanimate anaphoric demonstrative, referring to a previously mentioned
inanimate noun. The third, the indefinite pronoun, has a stem of {\ll ɦa-}, and
can refer to people or things in the abstract (\ie\ \qq{someone}), or introduce
new nouns thru gesturing or relative clauses:

\pex
\a
\begingl
\glpreamble hróŋgkeht weh ɦah zəj kálpkeht nájshəsdehrəwən. //
\gla hróŋg @ -kèht weh ɦah zəj kálp @ -keht nájshəsdeh @ -rə @ -wən //
\glb frighten -{\sc prs}.3p {\sc pn}.1.{\sc acc}.s {\sc ndef}.{\sc nom}.p {\sc
    rel} wish -{\sc prs}.3p die -{\sc subj} -{\sc pst}.1s //
\glft Those who wish me dead frighten me. //
\endgl

\a
\begingl
    \glpreamble həstekwò ɦahè. //
    \gla həs- @ tek @ -wò ɦahè //
    \glb not- {[care about]} -{\sc prs}.1s {\sc ndef}.{\sc acc}.s //
    \glft I don't care about that. //
\endgl
\xe

\begin{table}
\centering
\caption{Demonstrative Pronouns}
\label{tab:demonstrative pronouns}
\begin{tabular}{ll*{3}{>{\ll}c}}
    \toprule
    Pronoun & Case & \rm Singular & \rm Dual & \rm Plural \\ \midrule
    Animate    & Nominative   & kəs  & —     & kəh \\
               & Accusative   & keh  & kjə   & khəh \\
               & Instrumental & —    & —     & — \\
               & Genitive     & kaj  & ksa   & kajh \\
               & Locative     & kra  & kraj  & krah \\ \midrule
    Inanimate  & Nominative   & —    & —     & — \\
               & Accusative   & tóhe & tójə  & tóhəh \\
               & Instrumental & tója & tójaj & tójah \\
               & Genitive     & tój  & tóza  & tójh \\
               & Locative     & tóra & tóraj & tórah \\ \midrule
    Indefinite & Nominative   & ɦas  & ɦaj   & ɦah \\
               & Accusative   & ɦahè & ɦjə   & ɦəh \\
               & Instrumental & ɦja  & ɦjaj  & ɦjah \\
               & Genitive     & ɦaj  & ɦəza  & ɦajh \\
               & Locative     & ɦra  & ɦraj  & ɦrah \\
    \bottomrule
\end{tabular}
\end{table}

\paragraph{Reflexive Pronouns} In addition to the personal and demonstrative
pronouns, \Langname{} also has a reflexive pronoun, which is used to
refer back to the subject. Because it must refer back to the subject of the
sentence, it lacks nominative forms. It is also necessarily animate (since
inanimate nouns can't be subjects), so it lacks instrumental forms. It is
prototypically third person, although it can refer to first or second person
subjects thru coördination with the corresponding personal pronoun. The forms
of the reflexive pronouns are shown on \cref{tab:other pronouns}.

\begin{table}
\centering
\caption{Other Pronouns}
\label{tab:other pronouns}
\begin{tabular}{ll*{3}{>{\ll}c}}
    \toprule
    Pronoun & Case & \rm Singular & \rm Dual & \rm Plural \\ \midrule
    Reflexive     & Nominative   & —      & —       & — \\
                  & Accusative   & deh    & djə     & dəh \\
                  & Instrumental & —      & —       & — \\
                  & Genitive     & daj    & dza     & dajh \\
                  & Locative     & dra    & draj    & drah \\ \midrule
    Interrogative & Nominative   & dwətəs & dwətjə  & dwətəh \\
                  & Accusative   & dwəthe & dwətjə  & dwəthəh \\
                  & Instrumental & dwətja & dwətjaj & dwətjah \\
                  & Genitive     & dwətaj & dwətza  & dwətajh \\
                  & Locative     & dwətra & dwətraj & dwətrah \\
    \bottomrule
\end{tabular}
\end{table}

