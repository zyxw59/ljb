%&latex
\section{Nouns}

Nouns are generally formed by adding a nominalizing ending to a verbal root,
although some roots can be used on their own as nouns.

Nouns decline for three numbers: singular, dual, and plural, and six cases:
nominative, accusative, instrumental, genitive, locative, and intrative. The
nominative is only found in animate nouns, and serves as the subject of a verb.
The accusative serves as the object of a verb. The instrumental is only found
in inanimate nouns, and indicates the means of an action. The genitive can be
translated as \qq{from} or \qq{of}, and is used for motion away from, and
certain types of possession. The locative can be translated as \qq{to} or
\qq{at}, and indicates motion towards, location, other types of possession. In
the dual, the genitive and locative are replaced with a single case, the
intrative, which can be translated as \qq{between}. If a genitive or locative
meaning is specifically required, the plural form can be used. In addition to
their standalone meanings, different cases are used as the objects of
prepositions, and verbs may have specific uses for the genitive and locative
cases. The endings are given in \cref{tab:noun endings}.

Because inanimate nouns lack a nominative case, they can't be the subject of a
sentence. Instead, they are placed in the instrumental case, and the impersonal
form of the verb is used:

\ex
\begingl
    \glpreamble kónjòɦə ləmlárjà mə́whtə̀hjə̂. //
    \gla k´n @ -jò @ -ɦə lamlár @ -ja m.wh @ -tə̀h @ -jə̂ //
    \glb separate -{\sc ret} -{\sc prs}.0 mountain -{\sc ins}.s person -{\sc
        col} -{\sc acc}.d //
    \glft The mountain separates the two peoples. //
\endgl
\xe

\begin{table}[h]
\centering
\caption{Noun Endings}
\label{tab:noun endings}
\begin{threeparttable}
\makebox[\linewidth]{
\begin{tabular}{c*{3}{>{\ll}c}}
    \toprule
    Case & \rm Singular & \rm Dual & \rm Plural \\ \midrule
    Nominative   & (ə)s\tnote{\rm 1} & jə  & (ə)h\tnote{\rm 1} \\
    Accusative   & hè                & jə̂  & hə̂h \\
    Instrumental & ja                & jaj & jah \\
    Genitive     & (a)j\tnote{\rm 1} & —   & (a)jh\tnote{\rm 1} \\
    Locative     & râ                & —   & râh \\
    Intrative    & —                 & zâ  & — \\
    \bottomrule
\end{tabular}
}
\begin{tablenotes}
\item[1] Unlike unlike normal endings, these ending do not start with
    consonants. If they are added to an ending without a final consonant, their
    vowels are dropped.
\end{tablenotes}
\end{threeparttable}
\end{table}

\begin{table}[h]
\centering
\caption{Inflection of {\ll sohŋʷo} \qq{hand}, inanimate}
\begin{tabular}{c*{3}{>{\ll}c}}
    \toprule
    Case & \rm Singular & \rm Dual & \rm Plural \\ \midrule
    Accusative   & səhŋʷó-hə̀ & səhŋʷó-jə̀ & səhŋʷə-hə̂h \\
    Instrumental & səhŋʷə-jə & səhŋʷo-jə̂ & səhŋʷə-jəh \\
    Genitive     & səhŋʷəj   & —         & səhŋʷəjh \\
    Locative     & səhŋʷə-râ & —         & səhŋʷə-râh \\
    Intrative    & —         & səhŋʷə-zâ & — \\
    \bottomrule
\end{tabular}
\end{table}

\subsection{Pronouns}
\paragraph{Personal Pronouns} There are first and second person pronouns, as
well as several sets of demonstratives that function as third person pronouns.
Since subjects are marked on verbs, the nominative case of the pronouns are
only used when two subjects appear in comitative coördination (e.g. \dq{You and
    I \ldots}). The dual number cannot be used in this type of coördination,
and therefore the personal pronouns lack a nominative in the dual. The forms of
the pronouns are shown on \cref{tab:personal pronouns}.

\paragraph{Demonstratives} \Langname{} has three sets of demonstratives. The
first, with a stem of {\ll k.-}, is an animate anaphoric demonstrative, meaning
that it refers to an animate noun (generally a person or animal) that has
already been referred to. The second, with a stem of {\ll to-}, is an
inanimate anaphoric demonstrative, referring to a previously mentioned
inanimate noun. The third, the indefinite pronoun, has a stem of {\ll ɦa-}, and
can refer to people or things in the abstract (\ie\ \qq{someone}), or introduce
new nouns thru gesturing or relative clauses:

\pex
\a
\begingl
\glpreamble hréɠkèht wêh ɦah zəh kélpkèht néjshèsdəhrəwən. //
\gla hr´ɠ @ -kèht wêh ɦah zəj k´lp @ -kèht néjshèsdəh @ -rə @
-wən //
\glb frighten -{\sc prs}.3p {\sc pn}.1.{\sc acc}.s {\sc ndef}.{\sc nom}.p {\sc
    rel} wish -{\sc prs}.3p die -{\sc subj} -{\sc pst}.1s //
\glft Those who wish me dead frighten me. //
\endgl

\a
\begingl
    \glpreamble həstókwò ɦáhè. //
    \gla h.s- @ t.k @ -wò ɦáhè //
    \glb not- {[care about]} -{\sc prs}.1s {\sc ndef}.{\sc acc}.s //
    \glft I don't care about that. //
\endgl
\xe

\paragraph{Reflexive Pronouns} In addition to the personal and demonstrative
pronouns, \Langname{} also has a reflexive pronoun, which is used to
refer back to the subject. Because it must refer back to the subject of the
sentence, it lacks nominative forms. It is also necessarily animate (since
inanimate nouns can't be subjects), so it lacks instrumental forms. It is
prototypically third person, although it can refer to first or second person
subjects thru coördination with the corresponding personal pronoun. The forms
of the reflexive pronouns are shown on \cref{tab:other pronouns}.

\begin{table}[h]
\centering
\caption{Personal Pronouns}
\label{tab:personal pronouns}
\begin{tabular}{ll*{3}{>{\ll}c}}
    \toprule
    Person & Case & \rm Singular & \rm Dual & \rm Plural \\ \midrule
    First Person  & Nominative & wəs   & —     & thəh \\
                  & Accusative & wêh   & ʔéhè  & thə́hə̀h \\
                  & Genitive   & waj   & —     & thajh \\
                  & Locative   & wərâ  & —     & thərâh \\
                  & Intrative  & —     & ʔəzâ  & — \\ \midrule
    Second Person & Nominative & sə̂ns  & —     & sə̂h \\
                  & Accusative & sénhè & snə̂j  & sə́hə̀h \\
                  & Genitive   & sánàj & —     & sánàjh \\
                  & Locative   & sánrà & —     & sánràh \\
                  & Intrative  & —     & sánzà & — \\
    \bottomrule
\end{tabular}
\end{table}

\begin{table}
\centering
\caption{Demonstrative Pronouns}
\label{tab:demonstrative pronouns}
\begin{tabular}{ll*{3}{>{\ll}c}}
    \toprule
    Pronoun & Case & \rm Singular & \rm Dual & \rm Plural \\ \midrule
    Animate    & Nominative   & kəs  & —     & kəh \\
               & Accusative   & kêh  & kjə̂   & khə̂h \\
               & Instrumental & —    & —     & — \\
               & Genitive     & kaj  & —     & kajh \\
               & Locative     & krâ  & —     & krâh \\
               & Intrative    & —    & ksâ   & — \\ \midrule
    Inanimate  & Nominative   & —    & —     & — \\
               & Accusative   & tóhè & tójə̀  & tóhə̀h \\
               & Instrumental & tójò & tójòj & tójòh \\
               & Genitive     & tôj  & —     & tôjh \\
               & Locative     & tórò & —     & tóròh \\
               & Intrative    & —    & tózò  & — \\ \midrule
    Indefinite & Nominative   & ɦas  & ɦaj   & ɦah \\
               & Accusative   & ɦáhè & ɦjə̂   & ɦə̂h \\
               & Instrumental & ɦja  & ɦjaj  & ɦjah \\
               & Genitive     & ɦaj  & —     & ɦajh \\
               & Locative     & ɦrâ  & —     & ɦrâh \\
               & Intrative    & —    & ɦəzâ  & — \\
    \bottomrule
\end{tabular}
\end{table}

\begin{table}
\centering
\caption{Other Pronouns}
\label{tab:other pronouns}
\begin{tabular}{ll*{3}{>{\ll}c}}
    \toprule
    Pronoun & Case & \rm Singular & \rm Dual & \rm Plural \\ \midrule
    Reflexive     & Nominative   & —      & —       & — \\
                  & Accusative   & dêh    & djə̂     & də̂h \\
                  & Instrumental & —      & —       & — \\
                  & Genitive     & daj    & —       & dajh \\
                  & Locative     & drâ    & —       & drâh \\
                  & Intrative    & —      & dzâ     & — \\ \midrule
    Interrogative & Nominative   & dwətəs & dwətjə  & dwətəh \\
                  & Accusative   & dwéthè & dwətjə̂  & dwəthə̂h \\
                  & Instrumental & dwətjə & dwətjəj & dwətjəh \\
                  & Genitive     & dwətəj & —       & dwətəjh \\
                  & Locative     & dwətrâ & —       & dwətrâh \\
                  & Intrative    & —      & dwətzâ & — \\
    \bottomrule
\end{tabular}
\end{table}

