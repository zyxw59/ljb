\section{Phonology}
\begin{table}[h]
\centering
\caption{Consonants}
\label{tab:consonants}
\begin{tabu} to \textwidth {l*{4}{X[c]}X[2,c]*{6}{X[c]}}
    \toprule
    & \multicolumn{2}{c}{Labial} & \multicolumn{2}{c}{Alveolar} &
    \multicolumn{1}{c}{Palatal} & \multicolumn{2}{c}{Velar} &
    \multicolumn{2}{c}{Labiovelar} & \multicolumn{2}{c}{Glottal} \\
    \midrule
    Plosive     & p & b & t & d &   & k & ɡ & kʷ & ɡʷ & ʔ &   \\
    Implosive   &   & ɓ &   & ɗ &   &   & ɠ &    & ɠʷ &   &   \\
    Fricative   &   &   & s & z &   &   &   &    &    & h & ɦ \\
    Nasal       &   & m &   & n &   &   & ŋ &    & ŋʷ &   &   \\
    Approximant &   &   &   & r & j &   &   &    & w  &   &   \\
    Lateral     &   &   &   & l &   &   &   &    &    &   &   \\
    \bottomrule
\end{tabu}
\end{table}

\subsection{Roots}
\paragraph{Consonants}
Roots must follow the sonorance hierarchy, with more sonorant elements falling
closer to the nucleus than less sonorant elements. There are three categories
of consonants as follows, from most to least sonorant:
\begin{itemize}
    \item Approximants, laterals, and nasals (S)
    \item Fricatives (F)
    \item Plosives and implosives (P)
\end{itemize}
Roots are of the general form PFS–SFP, with an optional vowel separating the
onset and coda. Each category is optional, although the onset cannot be null.
If both a plosive and a fricative are present, they must agree in voice.

\paragraph{Vowels}
Roots do not have any inherent vowel associated with them. When a
root appears on it's own, the position of the vowel will be marked with a dot
(.) if the root is weak, or an acute accent (´) if it is strong. When it
appears with an ending, the root can appear in one of four forms, or grades:
\begin{itemize}
    \item {\ll ə}-grade
    \item {\ll e}-grade
    \item {\ll o}-grade
    \item {\ll a}-grade
\end{itemize}
The sonorants have syllabic allophones in the ə-grade. Syllabic \slashes{j} and
\slashes{w} will be denoted as {\ll i} and {\ll u}, while other syllabic
sonorants will be denoted as in the IPA.

\paragraph{Accent}
Roots or words do not have an inherent accent. Instead, each root (and ending)
is classified as {\em strong}, {\em weak}, or {\em grave}. Weak syllables
generally have no effect on the accent, but they are made strong if followed by
a grave syllable. The accent in a word falls on the first strong syllable. In
this document, strong syllables will be marked with an acute accent (´), and
grave syllables with a grave accent (\textgrave). Neutral syllables are
unmarked.
% TODO: add examples

\subsection{Endings}
Endings are of the same general form as roots, although they do have an
inherent grade. The grades of the root and the ending
are determined by \cref{tab:grades}. If there are multiple endings, only the
one closest to the root affects the grade of the root.
\begin{table}[h]
\centering
\caption{Grade ablaut}
\label{tab:grades}
\begin{tabular}{>{\ll}c l *{2}{>{\ll}c}}
    \toprule
    \rm Inherent grade & Position & \rm Root grade & \rm Ending grade \\
    \midrule
    ə & Always      & ə & ə \\ \midrule
    e & Accented    & e & e \\
      & Post-accent & — & e \\
      & Pre-accent  & e & ə \\
      & Else        & ə & ə \\ \midrule
    o & Accented    & o & o \\
      & Post-accent & — & o \\
      & Pre-accent  & o & ə \\
      & Else        & ə & ə \\ \midrule
    a & Accented    & a & a \\
      & Post-accent & — & о \\
      & Pre-accent  & e & e \\
      & Else        & ə & ə \\
    \bottomrule
\end{tabular}
\end{table}

