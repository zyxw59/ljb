\section{Phonology}
\begin{table}[h]
\centering
\caption{Consonants}
\label{tab:consonants}
\begin{tabu} to \textwidth {l*{4}{X[c]}X[2,c]*{6}{X[c]}}
    \toprule
    & \multicolumn{2}{c}{Labial} & \multicolumn{2}{c}{Alveolar} &
    \multicolumn{1}{c}{Palatal} & \multicolumn{2}{c}{Velar} &
    \multicolumn{2}{c}{Labiovelar} & \multicolumn{2}{c}{Glottal} \\
    \midrule
    Plosive     & p & b & t & d &   & k & ɡ & kʷ & ɡʷ & ʔ &   \\
    Implosive   &   & ɓ &   & ɗ &   &   & ɠ &    & ɠʷ &   &   \\
    Fricative   &   &   & s & z &   &   &   &    &    & h & ɦ \\
    Nasal       &   & m &   & n &   &   & ŋ &    & ŋʷ &   &   \\
    Approximant &   &   &   & r & j &   &   &    & w  &   &   \\
    Lateral     &   &   &   & l &   &   &   &    &    &   &   \\
    \bottomrule
\end{tabu}
\end{table}

\subsection{Syllables}
\paragraph{Consonants}
Syllables must follow the sonorance hierarchy, with more sonorant elements
falling closer to the nucleus than less sonorant elements. There are three
categories of consonants as follows, from most to least sonorant:
\begin{itemize}
    \item Approximants, laterals, and nasals (S)
    \item Fricatives (F)
    \item Plosives and implosives (P)
\end{itemize}
Syllables are of the general form PFS--SFP, with an optional vowel separating
the onset and coda. Each category is optional, although the onset cannot be
null. If both a plosive and a fricative are present, they must agree in voice.

\subsection{Accent}
The accent system of \Langname{} falls somewhere between a pitch accent and a
tone system. Each syllable can carry either a high tone ({\em acute\/}), a
falling tone ({\em grave\/}), or a neutral tone ({\em weak\/}). However, each
acute accent must be followed by a grave accent, and each grave accent must
follow an acute accent.

When a weak syllable follows an acute accent, it gains a grave accent.
Likewise, when a weak syllable precedes a grave accent, it gains an acute
accent. When an acute accent would follow another acute accent, or a grave
accent follow another grave accent, the sequence is replaced by an acute--grave
sequence.

\subsection{Vowels}
There are four phonemic vowels in \Langname{}: {\ll ə}, {\ll e}, {\ll o}, and
{\ll a}. Roots do not have any inherent vowel associated with them. When a
root appears on its own, the position of the vowel will be marked with a dot
(.) if the root is weak (no inherent accent), or an acute accent (´) if it is
strong (inherent acute accent). Endings do carry an inherent vowel. The vowel
of the first ending is know as the {\em stem vowel\/}. If the root has an acute
accent, it takes the stem vowel, otherwise it takes {\ll ə}.

A vowel with a grave accent changes depending on the stem vowel of the word,
according to \cref{tab:ablaut}. An unaccented vowel (neutral tone) is always
{\ll ə}.

\begin{table}[h]
\centering
\caption{Ablaut}
\label{tab:ablaut}
\begin{tabular}{*{3}{>{\ll}c}}
    \toprule
    \rm Vowel & \rm Stem vowel & \rm Grave form \\
    \midrule
    ə & \rm always & ə \\
    \midrule
    e & ə, o & ə \\
      & e, a & e \\
    \midrule
    o & ə, e & ə \\
      & o, a & o \\
    \midrule
    a & ə & ə \\
      & e & e \\
      & o & o \\
      & a & a \\
    \bottomrule
\end{tabular}
\end{table}

