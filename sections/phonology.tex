%&latex
\section{Phonology}
\begin{table}[h]
\centering
\caption{Consonants}
\label{tab:consonants}
\begin{tabu} to \textwidth {l*{4}{X[c]}X[2,c]*{6}{X[c]}}
    \toprule
    & \multicolumn{2}{c}{Labial} & \multicolumn{2}{c}{Alveolar} &
    \multicolumn{1}{c}{Palatal} & \multicolumn{2}{c}{Velar} &
    \multicolumn{2}{c}{Labiovelar} & \multicolumn{2}{c}{Glottal} \\
    \midrule
    Plosive           & p & b    & t & d    &   & k & ɡ    & kʷ & ɡʷ    & ʔ &   \\
    Prenasalized Stop &   & \^mb &   & \^nd &   &   & \^ŋɡ &    & \^ŋɡʷ &   &   \\
    Fricative         &   &      & s & z    &   &   &      &    &       & h & ɦ \\
    Nasal             &   & m    &   & n    &   &   & ŋ    &    & ŋʷ    &   &   \\
    Approximant       &   &      &   & r, l & j &   &      &    & w     &   &   \\
    \bottomrule
\end{tabu}
\end{table}

\subsection{Syllables}
\paragraph{Consonants}
Syllables must follow the sonorance hierarchy, with more sonorant elements
falling closer to the nucleus than less sonorant elements. There are three
categories of consonants as follows, from most to least sonorant:
\begin{itemize}
    \item Approximants and nasals
    \item Fricatives
    \item Plosives and prenasalized stops
\end{itemize}
Syllables consist of an onset of one or two consonants, followed by a vowel,
followed by zero to two consonants.
If both a plosive and a fricative are present, they must agree in voice.
Nasal consonants cannot come immediately before prenasalized stops in the same
syllable.

\subsection{Accent}
Words in \Langname{} can either have a \emph{mobile} accent, meaning the
location of the accent varies throughout the inflectional paradigm, or a
\emph{static} accent, meaning the location of the accent is fixed on a
particular syllable. In words with a mobile accent, the accent falls on the
last syllable with a vowel other than {\ll ə}. If no such syllable exists, the
word lacks an accent. In words with a static accent, it will be marked with an
acute accent (´).

\subsection{Vowels}
\label{sec:vowels}
There are four phonemic vowels in \Langname{}: {\ll ə}, {\ll e}, {\ll o}, and
{\ll a}. {\ll ə} can only occur in unaccented syllables; the rest can occur
anywhere.

\subsection{Orthography}
\Langname{} was never a written language, so we are free to use whatever
transcription scheme we find most convenient. This document will denote
consonants as listed on \cref{tab:consonants}, with the following exceptions:
\begin{itemize}
    \item {\ll g} and {\ll gʷ} instead of ɡ and ɡʷ
    \item {\ll mb nd ŋg ŋgʷ} instead of \^mb \^nd \^ŋɡ \^ŋɡʷ
\end{itemize}
Vowels will be transcribed as in \cref{sec:vowels}.

