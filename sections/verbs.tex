%&latex
\section{Verbs}
The basic form of a conjugated verb is stem + personal endings, where the stem
consists of a root with zero or more derivational endings.
The personal ending conveys person, number, and voice, while the derivational
endings provide aspect information.

\subsection{Aspect}
Each root is a associated with a default aspect.
To form other aspects, derivational suffixes are added.
Some of the most common suffixes are shown in \cref{tab:aspect suffixes}.
Because these suffixes are derivational in nature, they might also change the
meaning of the verb.
For example, the verb {\ll jan} \qq{to see} becomes {\ll jandéh} \qq{to notice}
in the inchoative aspect.

\begin{table}[h]
\centering
\caption{Aspect Suffixes}
\label{tab:aspect suffixes}
\begin{tabular}{>{\ll}l l}
    \toprule
    \multicolumn{1}{c}{Ending} & Aspect \\ \midrule
    -jo & Retrospective – the state resulting from an action \\
    -hemb & Habitual – an action repeated over a long period of time \\
    -ʔraŋgʷ & Continuous – an ongoing action \\
    -bet & Momentane – a single instance of an action \\
    -dəl & Aorist – a normally continuous action viewed as a whole \\
    -déh & Inchoative – the start of an action or state \\
    -kʷoz & Cessative – the end of an action or state \\
    \bottomrule
\end{tabular}
\end{table}

\subsection{Tense}
Verbs in \Langname{} are not explicitly marked for tense.
Instead, aspect plays a much more important role.
Perfective aspects, which refer to a single action, event, or point in time,
are implicitly past tense, since they can never occur in the exact moment of
the present – by the time the speaker has finished, the moment will have
passed.
Imperfective aspects, on the other hand, are more ambiguous as to tense.
Generally, they are assumed to refer to the present.
To specify a past tense meaning, context can be given thru adverbs or
perfective verbs, or a perfective aspect can be used instead.

\subsection{Personal Endings}
There are two sets of personal endings, one for the active voice, and one for
the passive voice.
Inanimate nouns do not have a nominative form, so there are no active endings
for third person inanimate.
The impersonal ending, {\ll -ɦàs}, is neither active nor passive in meaning.
The personal endings are shown on \cref{tab:verb personal endings}.

\begin{table}[h]
\centering
\caption{Verb Personal Endings}
\label{tab:verb personal endings}
\begin{tabu}spread 0pt{ll*{2}{>{\ll}l}}
    \toprule
    Person & Number &
    \multicolumn{1}{c}{Active} & \multicolumn{1}{c}{Passive} \\ \midrule
    Impersonal    & —        & \multicolumn{2}{c}{\ll -ɦàs} \\ \midrule
    First Person  & Singular & -wəs  & -wèh    \\
                  & Dual     & -ʔès  & -ʔèh    \\
                  & Plural   & -thəh & -thəhəh \\ \midrule
    Second Person & Singular & -sàns & -sànhè  \\
                  & Dual     & -snəj & -snəj   \\
                  & Plural   & -səh  & -səhəh  \\ \midrule
    Third Person  & Singular & -kəs  & -kəh    \\
    (Animate)     & Dual     & -kjə  & -kjə    \\
                  & Plural   & -kəh  & -khəh   \\ \midrule
    Third Person  & Singular & —     & -tòh    \\
    (Inanimate)   & Dual     & —     & -tòj    \\
                  & Plural   & —     & -tòhəh  \\ \bottomrule
\end{tabu}
\end{table}

\subsection{Mood}
\Langname{} has two moods, the indicative and the subjunctive.
The indicative is unmarked.
The subjunctive is marked by the suffix {\ll -rə}, inserted between the stem
and the personal ending.

The indicative mood is used when describing real facts or events in the present
or past.
The subjunctive is used in conditionals, imperatives, and as a future tense.

\subsection{Voice}
\Langname{} has two voices, active and passive, as well as a third form, the
impersonal.
The active voice is used when the subject of the verb is the willful agent of
the action.
Because the active voice requires volition, inanimate nouns can never be the
subject of an active verb.
In the active voice, the subject takes the nominative case.
The passive voice is used when the subject of the verb is the patient, target,
or experiencer of the action.
In the passive voice, the subject takes the accusative case.
The impersonal is used when there is neither a willful agent nor a specific
target of the action.
It is frequently used in weather expressions.
It is also used when there is a non-willful (particularly inanimate) agent.
In this usage, the agent takes the instrumental case if it is inamiate, or the
genitive case if it is animate.
However animate agents in impersonal constructions are rare; instead, they are
often replaced with an inanimate equivalent, or the active voice is used.
% TODO: add examples

\subsection{Negation}
Negation is expressed using the prefix {\ll həs}.

