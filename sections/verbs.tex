%&latex
\section{Verbs}
The basic form of a conjugated verb is stem + inflection, where the stem
consists of a root with zero or more derivational endings. The inflection
conveys person and number information, as well as some tense information, while
the derivational endings provide aspect information.

\subsection{Aspect}
Each root is a associated with a default aspect. To form other aspects,
derivational endings are added. Some of the most common endings are shown in
\cref{tab:aspect endings}. Because these endings are derivational in
nature, they might also change the meaning of the verb. For example, the verb
{\ll j.n} \qq{to see} becomes {\ll jen-déh} \qq{to notice} in the inchoative
aspect.

\begin{table}[h]
\centering
\caption{Aspect Endings}
\label{tab:aspect endings}
\begin{threeparttable}
\makebox[\linewidth]{
\begin{tabular}{>{\ll}c c}
    \toprule
    \rm Ending & Aspect \\ \midrule
    jò & Retrospective \\
    heɓ & Habitual \\
    ʔraɠʷ & Continuous \\
    bèt & Perfective\tnote{1} \\
    də́l & Perfective\tnote{2} \\
    déh & Inchoative \\
    hesdéh & Cessative \\
    \bottomrule
\end{tabular}
}
\begin{tablenotes}
\item[1] Used to form the perfective aspect from habitual verbs, referring to a
    single instance of the action.
\item[2] Used to form the perfective aspect from continuous verbs, referring to
    the complete action as a whole.
\end{tablenotes}
\end{threeparttable}
\end{table}

\subsection{Inflectional Endings}
There are three sets of inflectional endings. The first set of inflections are
used with plain roots in the present tense, or in the case of the perfective
aspects, the past tense. The second set are used after inflectional endings in
the present (or past perfective) tense. The third set are used with the
imperfective aspect to form the past tense. Each set of endings consists of ten
forms: nine for the first, second, and third person in the singular, dual, and
plural, and one for the impersonal. The forms of the endings are given by
\cref{tab:verb inflectional endings}.

\begin{table}[h]
\centering
\caption{Verb Inflectional Endings}
\label{tab:verb inflectional endings}
\begin{tabular}{ll*{3}{>{\ll}c}}
    \toprule
    Person & Number & \rm \parbox{2cm}{\centering Present Primary} & \rm
    \parbox{2cm}{\centering Present Secondary} & \rm Past \\ \midrule
    Impersonal    & —        & hà   & hə   & heh   \\ \midrule
    First Person  & Singular & wòn  & wən  & hon   \\
                  & Dual     & jès  & jə   & hej   \\
                  & Plural   & hèst & həs  & hes   \\ \midrule
    Second Person & Singular & sàn  & sən  & hens  \\
                  & Dual     & sàjs & səjs & hejs  \\
                  & Plural   & sàht & səh  & hes   \\ \midrule
    Third Person  & Singular & khèʔ & khəʔ & hek   \\
                  & Dual     & khèj & khəj & hékèj \\
                  & Plural   & khèh & khə  & hek   \\
    \bottomrule
\end{tabular}
\end{table}

\begin{table}[h]
\centering
\caption{{\ll j.n} \qq{to see} in a few tenses}
\begin{tabu} to \textwidth {ll*{3}{>{\ll}X[cm]}}
    \toprule
    Person & Number & \rm Present Imperfect & \rm Past Imperfect & \rm Past
    Inchoative (\qq{to notice}) \\ \midrule
    Impersonal & —        & ján-hà   & jən-həh   & jəndéh-hə̀   \\ \midrule
    First      & Singular & jón-wòn  & jən-hən   & jəndéh-wə̀n  \\
               & Dual     & jén-jès  & jən-həj   & jəndéh-jə̀   \\
               & Plural   & ján-hèst & jən-həs   & jəndéh-hə̀s  \\ \midrule
    Second     & Singular & ján-sàn  & jən-həns  & jəndéh-sə̀n  \\
               & Dual     & ján-sàjs & jən-həjs  & jəndéh-sə̀js \\
               & Plural   & ján-sàht & jən-həs   & jəndéh-sə̀h  \\ \midrule
    Third      & Singular & jén-khèʔ & jən-hək   & jəndéh-khə̀ʔ \\
               & Dual     & jén-khèj & jən-hékèj & jəndéh-khə̀j \\
               & Plural   & jén-khèh & jən-hək   & jəndéh-khə̀  \\
    \bottomrule
\end{tabu}
\end{table}

\subsection{Mood}
\Langname{} has two moods, the indicative and the subjunctive. The indicative
is unmarked. The subjunctive is marked in two different ways. On root verbs in
the present (or past perfective) tense, the subjunctive is marked by using
secondary endings, rather than primary ones. In all other cases, the
subjunctive is marked by the ending {\ll rə}, inserted between the stem and the
inflectional ending.
% TODO: add examples

\subsection{Passive Voice}
\Langname{} lacks a true passive voice. Instead, the impersonal is used, with
the normal object remaining in the accusative.
% TODO: add examples

\subsection{Negation}
Negation is expressed using the prefix {\ll həs}.

