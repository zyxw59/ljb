\section{Particles}
Particles are roots which are used on their own, without inflectional endings.
Unlike verbal and nominal roots, particle roots have an intrinsic grade.

\subsection{Numbers}
Numbers are placed before the nouns they modify. They can also take the
derivational ending {\ll ʔej} to form an ordinal, which acts as an imperfective
verb stem. However, \qq{first}, \qq{second}, and \qq{fifth} are suppletive. The
system has both decimal and quinary components, as seen in the forms for
numbers six through nine, as well as 15, 25, and 50. The numbers are shown in
\cref{tab:numbers}. Numbers beyond 50 are formed differently in various
daughter languages.
% between 50 and 75 are formed with {\ll khen-sah}
% plus {\ll ʔə} \qq{and} plus the remainder. Numbers between 75 and 100 are
% formed similarly with {\ll tjeh-sah-sah}. 

\begin{table}[h]
\centering
\caption{Numbers}
\label{tab:numbers}
\begin{threeparttable}
\makebox[\linewidth]{
\begin{tabular}{c*{2}{>{\ll}c}}
    \toprule
    Number & \rm Root & \rm Ordinal\tnote{1} \\ \midrule
    1  & rn̩            & r´w \\
    2  & li            & ŋʷ.k \\
    3  & tjeh          & \\
    4  & pət           & \\
    5  & sah           & dlow-jò \\
    6  & sah-rn̩        & \\
    7  & sah-li        & \\
    8  & sah-tjeh      & \\
    9  & sah-pət       & \\
    10 & khen          & \\
    11 & khen-rn̩       & \\
    \vdots & \vdots    & \\
    15 & tjeh-sah      & tjeh-dlow-jò \\
    16 & tjeh-sah-rn̩   & \\
    \vdots & \vdots    & \\
    20 & li-khen       & \\
    25 & sah-sah       & sah-dlow-jò \\
    30 & tjeh-khen     & \\
    35 & tjeh-khen-sah & \\
    40 & pət-khen      & \\
    45 & pət-khen-sah  & \\
    50 & khen-sah      & khen-dlow-jò \\
    % 75 & tjeh-sah-sah  & tjeh-sah-dlow-jò \\
    100 & gʷew         & \\
    \bottomrule
\end{tabular}
}
\begin{tablenotes}
\item[1] Only shown if irregular.
\end{tablenotes}
\end{threeparttable}
\end{table}

\subsection{Prepositions and Adverbs}
Prepositions are placed before nouns to form indirect objects as well as
temporal and spacial modifiers for verbs. Each preposition governs one or more
cases. Prepositions which govern multiple cases generally take on distinct
meanings for each case. Adverbs are like prepositions, except they lack an
object, and modify the verb on their own.

\subsection{Conjunctions}
Conjunctions join two elements of a sentence. They generally only join elements
of the same type (i.e. two nouns or two verbs). The main exception to this rule
is the conjunction {\ll zi}, which joins a noun to a relative clause.
% TODO: fill out, or delete entirely

