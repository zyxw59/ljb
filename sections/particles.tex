%&latex
\section{Particles}
Particles are roots which are used on their own, without inflectional endings.
Unlike verbal and nominal roots, particle roots can exist as stand-alone words,
and have intrinsic vowels.

\subsection{Numbers}
Numbers one thru four are invariant. {\ll sah}, acts like a noun, and can be
translated as \qq{a group of five}. It agrees with the head noun in case, tho
in numbers five thru nine, it is singular, while the head noun is plural. This
construction is an instance of attributive coördination. Numbers six thru nine
are formed with {\ll sah}, followed by a number one thru four, followed by the
noun. The noun agrees in number with the last numeral, so six takes a singular
noun, and seven takes a dual noun. When such a noun phrase is the subject of a
verb, the verb is still plural.

{\ll khen}, \qq{a group of ten}, acts the same as {\ll sah}. Fifteen is {\ll
    tjeh sah}, with {\ll sah} in the plural; twenty is {\ll ləj khen}, with
{\ll khen} in the dual; twenty five is {\ll sah sah}, with the first {\ll sah}
in the singular, and the second in the plural. In general, numbers ending in
0--4 are formed with the number of tens, followed by an inflected form of {\ll
    khen}, followed by the ones digit; numbers ending in 5--9 are formed with
the number of fives, followed by an inflected form of {\ll sah}.

\begin{table}[h]
\centering
\caption{Invariant Numbers}
\label{tab:numbers}
\begin{tabular}{c>{\ll}c}
    \toprule
    Number & \rm Word \\ \midrule
    1  & rən \\
    2  & ləj \\
    3  & tjeh \\
    4  & pət \\
    \bottomrule
\end{tabular}
\end{table}

\subsection{Prepositions and Adverbs}
Prepositions are placed before nouns to form indirect objects as well as
temporal and spacial modifiers for verbs. Each preposition governs one or more
cases. Prepositions which govern multiple cases generally take on distinct
meanings for each case. Adverbs are like prepositions, except they lack an
object, and modify the verb on their own.

\subsection{Conjunctions}
Conjunctions join two elements of a sentence. They generally only join elements
of the same type (i.e. two nouns or two verbs). The main exception to this rule
is the conjunction {\ll zəj}, which joins a noun to a relative clause.
% TODO: fill out, or delete entirely

