%&latex
\section{Particles}
Particles are roots which are used on their own, without inflectional endings.
Unlike verbal and nominal roots, particle roots can exist as stand-alone words,
and have intrinsic vowels.

\subsection{Numbers}
Numbers are placed before the nouns they modify. They can also take the
derivational ending {\ll ʔèj} to form an ordinal, which acts as an imperfective
verb stem. However, \qq{first}, \qq{second}, and \qq{fifth}, as well as the
ordinal forms of numbers ending in {\ll rən}, {\ll ləj}, or {\ll sah}, are
suppletive. The system has both decimal and quinary components, as seen in the
forms for numbers six through nine, as well as 15, 25, and 50. The numbers are
shown in \cref{tab:numbers}. Numbers beyond 50 are formed differently in
various daughter languages.
% between 50 and 75 are formed with {\ll khen-sah}
% plus {\ll ʔə} \qq{and} plus the remainder. Numbers between 75 and 100 are
% formed similarly with {\ll tjeh-sah-sah}. 

\begin{table}[h]
\centering
\caption{Numbers}
\label{tab:numbers}
\begin{threeparttable}
\makebox[\linewidth]{
\begin{tabular}{c*{2}{>{\ll}c}}
    \toprule
    Number & \rm Root & \rm Ordinal\tnote{1} \\ \midrule
    1  & rən          & r´w \\
    2  & ləj          & ŋʷ.k \\
    3  & tjeh         & \\
    4  & pət          & \\
    5  & sah          & dlówjò \\
    6  & sáhrə̀n       & sah r´w \\
    7  & sáhlə̀j       & sah ŋʷ.k \\
    8  & sáhtjèh      & \\
    9  & sáhpə̀t       & \\
    10 & khen         & \\
    11 & khénrə̀n      & khen r´w \\
    \vdots & \vdots   & \\
    15 & tjéhsàh      & tjəhdlówjò \\
    16 & tjéhsàh rən  & tjéhsàh r´w \\
    \vdots & \vdots   & \\
    20 & lə́jkhèn      & \\
    25 & sáhsàh       & səhdlówjò \\
    30 & tjéhkhèn     & \\
    35 & tjéhkhèn sah & tjéhkhèn dlówjò \\
    40 & pə́tkhèn      & \\
    45 & pə́tkhèn sah  & pə́tkhèn dlówjò \\
    50 & khénsàh      & khəndlówjò \\
    100 & gʷew        & \\
    \bottomrule
\end{tabular}
}
\begin{tablenotes}
\item[1] Only shown if irregular.
\end{tablenotes}
\end{threeparttable}
\end{table}

\subsection{Prepositions and Adverbs}
Prepositions are placed before nouns to form indirect objects as well as
temporal and spacial modifiers for verbs. Each preposition governs one or more
cases. Prepositions which govern multiple cases generally take on distinct
meanings for each case. Adverbs are like prepositions, except they lack an
object, and modify the verb on their own.

\subsection{Conjunctions}
Conjunctions join two elements of a sentence. They generally only join elements
of the same type (i.e. two nouns or two verbs). The main exception to this rule
is the conjunction {\ll zəj}, which joins a noun to a relative clause.
% TODO: fill out, or delete entirely

