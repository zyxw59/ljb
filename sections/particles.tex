%&latex
\section{Particles}
Particles are roots which are used on their own, without inflectional endings.
Unlike verbal and nominal roots, particle roots can exist as stand-alone words.

\subsection{Numbers}
Numbers one thru four are invariant, and are shown on \cref{tab:numbers}. {\ll
    sah}, acts like a noun, and can be translated as \qq{a group of five}. It
agrees with the head noun in case, tho in numbers five thru nine, it is
singular, while the head noun is plural. This construction is an instance of
attributive coördination. Numbers six thru nine are formed with {\ll sah},
followed by a number one thru four, followed by the noun. The noun agrees in
number with the last numeral, so six takes a singular noun, and seven takes a
dual noun. When such a noun phrase is the subject of a verb, the verb is still
plural.

Numbers greater than nine are formed in various ways at different points in
time, and in different areas. There are three systems: the pure quinary system,
the pure decimal system, and the mixed system.

\paragraph{Quinary System} In the quinary system, numbers greater than nine are
formed with the number of fives, followed by an inflected form of {\ll sah},
followed by the remainder (one thru four). Ten is {\ll ləj sah}, with {\ll sah}
in the dual; fifteen is {\ll tjeh sah}, with {\ll sah} in the plural; and so
on. Twenty five is {\ll sah sah}, with the first {\ll sah} in the singular, and
the second one in the plural. Exact numbers beyond fifty or so are uncommon,
and {\ll gʷew} (which functions like a noun similarly to {\ll sah}) can refer
to any arbitrarily large number.

\paragraph{Decimal System} In the decimal system, {\ll khen}, translatable as
\qq{a group of ten}, acts like {\ll sah}. Numbers greater than nine are formed
with the number of tens, followed by an inflected form of {\ll khen}, followed
by the ones digit, formed identically to one thru nine. {\ll gʷew} specifically
means a hundred.

\paragraph{Mixed System} The mixed system takes elements from both the decimal
system and the quinary system. For small numbers (usually less than 30, tho the
exact cut-off can vary), the decimal system is used for numbers ending in 0--4,
while the quinary system is used for numbers ending in 5--9. For larger
numbers, the decimal system is used.

\begin{table}[h]
\centering
\caption{Invariant Numbers}
\label{tab:numbers}
\begin{tabular}{c>{\ll}c}
    \toprule
    Number & \rm Word \\ \midrule
    1  & rən \\
    2  & ləj \\
    3  & tjeh \\
    4  & pət \\
    \bottomrule
\end{tabular}
\end{table}

\subsection{Prepositions and Adverbs}
Prepositions are placed before nouns to form indirect objects as well as
temporal and spacial modifiers for verbs. Each preposition governs one or more
cases. Prepositions which govern multiple cases generally take on distinct
meanings for each case. Adverbs are like prepositions, except they lack an
object, and modify the verb on their own.

\subsection{Conjunctions}
Conjunctions join two elements of a sentence. They generally only join elements
of the same type (i.e. two nouns or two verbs). The main exception to this rule
is the conjunction {\ll zəj}, which joins a noun to a relative clause.
% TODO: fill out, or delete entirely

