-bèt & Perfective aspect of habitual verbs, \ie\ a single instance of an action. After a number, refers to a number of instances of the action \\
-déh & Inchoative aspect – entering into a state, or the start of an action \\
-də̀l & Perfective aspect of continuous verbs, \ie\ the action as a whole \\
-dwens & Benefactive – an action done on behalf of someone or something. The object of the original verb is placed in the genitive, and the benefactor in the accusative. \\
-hemb & Habitual aspect \\
-hesdéh & Cessative aspect – exiting a state, or the end of an action \\
-jed & Intensive; superlative \\
-jò & Retrospective aspect – the state resulting from an action \\
-ʔraŋgʷ & Continuous aspect \\
-tjór & Causative – causing someone or something to perform an action or enter a state. The subject of the original verb is placed in the accusative, and the object of an originally transitive verb is dropped or placed in the genitive or locative case, depending on the verb. The object of an impersonal verb remains in the accusative. \\
