-bet & Momentane aspect – a single instance of an action \\
-déh & Inchoative aspect – entering into a state, or the start of an action \\
-dəl & Aorist aspect – a normally continuous action viewed as a whole \\
-dwens & Benefactive – an action done on behalf of someone or something. The object of the original verb is placed in the genitive, and the benefactor in the accusative. \\
-hemb & Habitual aspect – an action repeated over a long period of time \\
-jed & Intensive; superlative \\
-jo & Retrospective aspect – the state resulting from an action \\
-kʷoz & Cessative aspect – the end of an action or state \\
-ʔraŋgʷ & Continuous aspect – an ongoing action \\
-tjór & Causative – causing someone or something to perform an action or enter a state. The subject of the original verb is placed in the accusative, and the object of an originally transitive verb is dropped or placed in the genitive or locative case, depending on the verb. The object of an impersonal verb remains in the accusative. \\
